% !TeX root = ../example.tex

\makeatletter
% 申请学位的学科门类:小二号宋体字,字距延伸0.5bp,所以\CJKglue应该设为1 bp。
% 修改:太原理工大学学科门类:学历:初号楷体,门类:20bp楷体 楷体均加粗,所以此处使用伪粗楷
% 学位名称
\newcommand\tyut@titlepage@degree{%
	\thu@stretch{11.3cm}{\bfseries\chuhao\kaishu%使用模板定义的弹性盒子
		\ifthu@degree@doctor
			博士%
		\else
			\ifthu@degree@master
				硕士%
			\fi
		\fi%
		学位论文%
	}%
}%

% 学位类型
%\newcommand\tyut@titlepage@degree@type{%
%	\begingroup
%	\CJKfamily+{} % \CJKfamily+ 对所有字符均有效,详见xeCJK
%%\xiaoer
%%	\def\CJKglue{\hskip 1bp}%
%	\bfseries\kaishu\fontsize{20bp}{31.2bp}\selectfont
%	\ifthu@degree@master
%	\ifthu@degree@type@professional
%	(专\ 业\  %
%	\else
%	\ifthu@degree@type@academic
%	(学\ 术\  %
%	\fi
%	\fi
%	学\ 位)
%	\fi
%	\endgroup
%}

\newcommand\tyut@titlepage@degree@type{%
	\centering
	\thu@stretch{4.8cm}{\bfseries\fontsize{20bp}{31.2bp}\selectfont\kaishu%
		\ifthu@degree@master
			\ifthu@degree@type@professional
				(专业%
			\else
				\ifthu@degree@type@academic
					(学术%
				\fi
			\fi%
			学位)%
		\fi%
	}%
}%

% 标题页作者信息表
\newcommand\tyut@titlepage@info@tabular[4]{%
	\def\tyut@info@item##1##2{%
		\thu@pad{#1}{\thu@fixed@box{#2}{##1}}%
		%      \thu@pad{#3}{:}% 修改:太原理工大学要求姓名信息居中,且需要添加下划线,\smash可以得到更接近文字的下划线。
		% 可以选择用\CJKunderline[thickness=<width>]{contents}来调整线宽
		\underline{\smash{\makebox[6cm][c]{##2}}} \\ 
	}%
	\begin{tabular}{l}%
		\renewcommand\arraystretch{1}%
		#4%
	\end{tabular}%
}

% 具体列表:可以自己添加
\newcommand\tyut@titlepage@info{%
	\tyut@titlepage@info@tabular{2.8cm}{2.82cm}{0cm}{%
		\tyut@info@item{姓名:}{\thu@author}%
		\tyut@info@item{学号:}{2000123456}%
		\tyut@info@item{培养单位:}{\thu@department}%
		\tyut@info@item{专业领域:}{计算机技术}%
		\tyut@info@item{研究方向:}{硕士论文}%
		\tyut@info@item{导师姓名:}{XXX \quad 讲师}%
		\tyut@info@item{企业导师:}{XXX \quad 教授}%
	}\par
}

% 论文成文打印的日期,用三号宋体汉字,字距延伸0.5bp,所以\CJKglue应该设为1 bp
\newcommand\tyut@titlepage@date{%
	\begingroup
	\kaishu\sanhao
	% 修改:太原理工大学要求三号楷体,阿拉伯数字日期,不做字距延伸
	论文提交日期:\thu@format@date{\thu@date@zh@digit@short}{\thu@date}\par
	\endgroup
}

% 英文封面格式
\newcommand\tyut@titlepage@en@title{%题目
	\begingroup
	\bfseries\xiaoer\selectfont
	\thu@title@en\par
	\endgroup
}
\newcommand\tyut@thesis@name@en{%
	\ifthu@degree@master
	Thesis%
	\else
	Dissertation%
	\fi
}

\newcommand\tyut@titlepage@en@date{%日期 修改:太原理工大学要求newtimes字体四号行距1.5
	\begingroup
	\sihao[27.4bp]
	\thu@format@date{\thu@date@en@short}{\thu@date}\par
	\endgroup
}

% 中文封面
\newcommand\tyut@titlepage@cn{
	\newgeometry{
		top     = 2.54cm,
		bottom  = 2.54cm,
		hmargin = 3.18cm,
	}%
	
	\thispagestyle{empty}%
	\thusetup{language = chinese}%
	\begingroup % 修改:太原理工大学,分类号,学校代码,密级在顶部左右两侧,小四字体
	{\kaishu\xiaosi%
		\setlength{\parindent}{0em}
		分类号:{\fontsize{11pt}{13.2pt}\thu@clc}\hfill
		学校代码:\makebox[1.35cm][l]{{\fontsize{11pt}{13.2pt}10112}{\enspace}}\par
		\hfill 密\quad\enspace 级:\makebox[1.35cm][l]{\thu@titlepage@secret} \par
	}%
	%    \parbox[t][2cm][t]{\textwidth}{%
		%      \hskip -21.5pt%
		%      \thu@titlepage@secret
		%    }\par
	\centering % logo,学历
	\vskip 34.32pt%
	\parbox{\textwidth}{
		\centering
		\includegraphics[height=2.62cm,width=10.79cm]{logo.png}
	}
	\vskip 12pt%
	\tyut@titlepage@degree
	\vskip 16pt
	\quad\tyut@titlepage@degree@type
	\vskip 68.64pt%
	%	\vfill
	% 题目
	\begingroup
	\sffamily\fontsize{22bp}{46.8bp}\selectfont
	\bfseries\kaishu \erhao\thu@title\par
	\endgroup
	\ifthu@main@language@english
	\vskip 5.4pt%
	\begingroup
	\sffamily\bfseries\fontsize{20bp}{31.2bp}\selectfont
	\thu@title@en\par
	\endgroup
	\vskip -9.2pt%
	\fi
	%    \vskip 34.1pt%
	\vfill
	\parbox[t][7.25cm][t]{\textwidth}{% 
		% 修改:太原理工大学作者及导师信息部分使用三号楷体
		% 行距需要手动调整
		\centering
		\kaishu\sanhao[30.5bp]\selectfont % 此处为字体字号
		\tyut@titlepage@info % 此处为作者导师信息
	}\par
	\vskip 80bp
	\parbox[t][0.6cm][t]{\textwidth}{\centering\tyut@titlepage@date}\par
	\vskip 16bp
	\endgroup
	\clearpage
	\restoregeometry
	\thu@reset@main@language
}
% 英文封面
\newcommand\tyut@titlepage@en{
	\newgeometry{
		top     = 2.54cm,
		bottom  = 2.54cm,
		hmargin = 3.18cm,
	}%
	\thispagestyle{empty}%
	\thusetup{language = english}%
	\begingroup%
	\centering%
	\null\vskip 8pt%
	\begingroup
	\fontsize{15bp}{31.5bp}\selectfont A Dissertation Submitted to
	\\
	Taiyuan University of Technology
	\\
	In partial fulfillment of the requirement
	\\
	For the degree of %
	\ifthu@degree@doctor
	Doctor
	\else
	\ifthu@degree@master
	Master
	\fi
	\fi%
	\\
	\endgroup
	\vskip 4.6cm%
	\tyut@titlepage@en@title
	\vskip 49.92bp%
	\sihao By\par%
	\sihao[27.3bp]\thu@author@en\par%
	\vfill
	\sihao[27.3bp]\thu@department@en\\
	\vskip 3pt%
	\tyut@titlepage@en@date
	\vskip 22.5pt%
	\endgroup
	\thu@reset@main@language
	\clearpage
	\restoregeometry
}
\makeatother

\begin{titlepage}
	\makeatletter
	\cleardoublepage
	\thu@pdfbookmark{-1}{\thu@title}%
	\tyut@titlepage@cn
	\cleardoublepage
	\tyut@titlepage@en
	\cleardoublepage
	\makeatother
\end{titlepage}
