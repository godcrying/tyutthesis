% !TeX root = ./example.tex

% 论文基本信息配置

\thusetup{
  %******************************
  % 注意:
  %   1. 配置里面不要出现空行
  %   2. 不需要的配置信息可以删除
  %   3. 建议先阅读文档中所有关于选项的说明
  %******************************
  %
  % 输出格式
  %   选择打印版(print)或用于提交的电子版(electronic),
  % 前者会插入空白页并设置边距以便直接双面打印
  % 后者将页面居中,电子版查看更美观。
  output = print,
  % 设置是否去掉页眉页脚,一般用于查重版本。默认为false。
  paper-check = false,
  %
  % 标题
  %   可使用“\\”命令手动控制换行
  %
  title  = {基于\LaTeX 的毕业论文编写\\方法的研究},
  title* = {The Research on writting method\\ of Dissertation Based on \\ \LaTeX },
  %
  %
  % 培养单位
  %   填写所属院系的全名
  %
  department = {信息与计算机学院},
  department* = {College of Information and Computer},
  %
  % 姓名
  author  = {李小明},
  author* = {Xiaoming Li},
  % 学位 
  degree = master,
  % 学位类型
  degree-type = professional,
  %
  % 日期
  %   使用 ISO 格式;默认为当前时间
  %
  % date = {2019-07-07},
  %
  % 密级和年限
  %   秘密, 机密, 绝密
  %
  % secret-level = {秘密},
  % secret-year  = {10},
  %
  % 博士后专有部分 
  %
  % clc                = {xxx}, % 分类号
  % udc                = {UDC},
  % id                 = {编号},
  % discipline-level-1 = {计算机科学与技术},  % 流动站(一级学科)名称
  % discipline-level-2 = {系统结构},          % 专业(二级学科)名称
  % start-date         = {2011-07-01},        % 研究工作起始时间
}

% 载入所需的宏包
\usepackage{subfig}
\DeclareSubrefFormat{parens}{#1\,(#2)} 
\captionsetup[subfloat]{subrefformat=parens} % 修改:引用子图要有括号。

% 修改:太原理工大学:添加算法环境
\usepackage{algorithm}
\usepackage{algorithmicx}
\usepackage{algpseudocode}
\renewcommand{\algorithmicrequire}{\textbf{输入:}}  
\renewcommand{\algorithmicensure}{\textbf{输出:}} 
% 定义算法字体为五号。自己按情况修改。
\makeatletter
\algrenewcommand\ALG@beginalgorithmic{\wuhao}
\makeatother


%\usepackage{setspace} % 与cls文件中的\@floatboxreset冲突
% 定理类环境宏包
\usepackage{amsthm}
% 也可以使用 ntheorem
% \usepackage[amsmath,thmmarks,hyperref]{ntheorem}
\usepackage{mathtools}

\thusetup{
  %
  % 数学字体
  math-style = GB,  % GB | ISO | TeX
  math-font  = xits,  % sitx | xits | libertinus
  integral-limits = false, % 积分上下限放置在右侧
}

% 可以使用 nomencl 生成符号和缩略语说明
% \usepackage{nomencl}
% \makenomenclature

% 表格加脚注
\usepackage{threeparttable}

% 表格中支持跨行
\usepackage{multirow}

% 固定宽度的表格。
% \usepackage{tabularx}

% 跨页表格
\usepackage{longtable}
% 单元格
\usepackage{makecell}

% 量和单位
\usepackage{siunitx}

% 参考文献使用 BibTeX + natbib 宏包
% 顺序编码制
\usepackage[sort]{natbib}
\bibliographystyle{thuthesis-numeric}
%\bibliographystyle{gbt7714-numerical}
% 著者-出版年制
% \usepackage{natbib}
% \bibliographystyle{thuthesis-author-year}

% 本科生参考文献的著录格式
% \usepackage[sort]{natbib}
% \bibliographystyle{thuthesis-bachelor}

% 参考文献使用 BibLaTeX 宏包
% \usepackage[backend=biber,style=thuthesis-numeric]{biblatex}
% \usepackage[backend=biber,style=thuthesis-author-year]{biblatex}
% \usepackage[backend=biber,style=apa]{biblatex}
% \usepackage[backend=biber,style=mla-new]{biblatex}
% 声明 BibLaTeX 的数据库
% \addbibresource{ref/refs.bib}

% 定义所有的图片文件在 figures 子目录下
\graphicspath{{figures/}}

% 数学命令
\makeatletter
\newcommand\dif{%  % 微分符号
  \mathop{}\!%
  \ifthu@math@style@TeX
    d%
  \else
    \mathrm{d}%
  \fi
}
\makeatother

% hyperref 宏包在最后调用
\usepackage{hyperref}
\usepackage{color}
\usepackage{rotfloat} % 图片横排
\usepackage[export]{adjustbox} % 图片顶端对齐
% 写作时参考文献位置占位
\newcommand*{\needcite}{\color{red}[cite]}